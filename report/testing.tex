\newcommand{\exsetup}[6]{%
\begin{table}[htbp]
    \centering
    \begin{tabular}{lll}
        \toprule
        \multirow{2}{*}{Best arm}   & $\mu$    & #1 \\
                                    & $\sigma$ & #2 \\
        \multirow{2}{*}{Other arms} & $\mu$    & #3 \\
                                    & $\sigma$ & #4 \\
        \multicolumn{2}{l}{Number of arms}     & #5 \\
        \bottomrule
    \end{tabular}
    \caption{Simulation settings for figure~\ref{#6}}
\end{table}} 

\newcommand{\ob}{$\sigma_{ob}$}

\chapter{Results}
\label{ch:results}

Here we present a representative subset of the obtained results.
We provide graphs to visualise how different parameters influence the results.

Parameter values are likely to interact and impact the end result in different ways, and it may be difficult to capture these interactions directly.
By fixing all but one parameter and testing for a range of values for this parameter, we capture different aspects of how the optimal observation noise changes.

\section{Multiarmed Bandit}
Figure~\ref{fig:ex1} illustrates why, when using LTS, it is important to make a good choice of observation noise.
In this case we see that the total regret achieved increases rapidly when the value is set too low, while it increases slightly as we move past the optimum.
With different setups we see similar results, although the slope after the optimum may be steeper or closer to zero.
\begin{figure}[htbp]
    \begin{gnuplot}[terminal=epslatex,terminaloptions=color]
    set style data lines
    set grid
    set xlabel "Observation noise"
    set ylabel "Total regret"
    set key bottom
    plot "../data/bruteforce/good-5.0,2.0\_bad-4.0,4.0\_est-10.0,0.5\_num-2\_obstart-0.15\_obend-0.95\_obstep-0.01\_rounds-10000\_reps-10000.data" using 2:(column(1)==1000?(5000-column(3)):(1/0)) title ""
\end{gnuplot}
\caption{Resulting total regret from varying observation noise.}
\label{fig:ex1}
\end{figure}
\exsetup{5.0}{2.0}{4.0}{4.0}{2}{fig:ex1}

\begin{figure}[htbp]
    \begin{gnuplot}[terminal=epslatex,terminaloptions=color]
    set style data lines
    set grid
    set xlabel "Observation noise"
    set ylabel "Total regret"
    set log x
    set xrange [4:1000]
    plot "../data/instantrewards/good-5.0,2.0\_bad-4.0,4.0\_est-10.0,0.5\_num-2\_obnoise-0.51\_rounds-1000\_reps-100000\_algo-LTS.data" using 1:(5-column(2)) title "\\textsc{lts: 0.51}", "../data/instantrewards/good-5.0,2.0\_bad-4.0,4.0\_est-10.0,0.5\_num-2\_obnoise-0.6\_rounds-1000\_reps-100000\_algo-LTS.data" using 1:(5-column(2)) title "\\textsc{lts: 0.60}", "../data/instantrewards/good-5.0,2.0\_bad-4.0,4.0\_num-2\_rounds-1000\_reps-100000\_algo-UCB1.data" using 1:(5-column(2)) title "\\textsc{ucb1}"
\end{gnuplot}
\caption{Instant regret for UCB1 and LTS.}
\label{fig:ex2}
\end{figure}
\exsetup{5.0}{2.0}{4.0}{4.0}{2}{fig:ex2}

LTS is superior to UCB1, when the appropriate observation noise is selected.
In figure~\ref{fig:ex2} this superiority is clearly demonstrated:
Even though UCB1 performs quite good from the start it seems to stop reducing its regret after a while, whereas LTS converges towards zero regret.
The figure also illustrates how a suboptimal choice of \ob{} causes the algorithm to perform worse.

\begin{figure}[hbtp]
    \centering
    \begin{gnuplot}[terminal=epslatex,terminaloptions=color]
    set style data lines
    set grid
    set hidden3d
    set pm3d
    set xlabel "Arms" rotate
    set ylabel "Rounds" rotate
    set zlabel "Observation noise" rotate
    set log xy
    set xyplane 0
    set xtics 2
    set yrange [10:1000]
    set zrange [0.01:0.25]
    set palette rgbformulae 23,28,3
    splot "../data/bandit/good-5.0,0.5\_bad-4.0,0.5\_est-10.0,2.0\_num-2,5,10,50\_rounds-1000\_algo-UCB1.data" using 5:6:7 title ""
    \end{gnuplot}
\caption{Best observation noise varying with the number of arms. Values for when the number of arms is greater than rounds are not included.}
\label{fig:ex3}
\end{figure}
\exsetup{5.0}{0.5}{4.0}{0.5}{Variable}{fig:ex3}

We see generally that as the number of rounds increases, so does the optimal \ob{}.
This is especially apparent in figure~\ref{fig:ex3}, where we have varied the number of arms available to the player.
As the numer of possible actions increases, the optimal \ob{} decreases.
It makes intuitive sense because if there is a large number of arms then we need to be more greedy in our choice of arms, as most of the arms pulled for exploration’s sake will be suboptimal.

\begin{figure}[htbp]
    \centering
    \begin{gnuplot}[terminal=epslatex,terminaloptions=color]
    set style data lines
    set grid
    set hidden3d
    set pm3d
    set ylabel "Bad arm mean" rotate
    set xlabel "Rounds" rotate
    set zlabel "Observation noise" rotate
    set log x
    set xyplane 0
    set ytics 0.5
    set xrange [100:1000]
    set yrange [3:4.5]
    set view 40,36
    set palette rgbformulae 23,28,3
    splot "../data/bandit/good-5.0,0.5\_bad-(0.5,4.5,0.5),0.5\_est-10.0,2.0\_num-5\_rounds-1000\_algo-UCB1.data" using 6:3:7 title ""
    \end{gnuplot}
\caption{Best observation noise varying with the mean on the suboptimal arms. As the mean is lowered past 3 or so changing observation noise stops having an effect, so these numbers are excluded from this figure.}
\label{fig:ex4}
\end{figure}
\exsetup{5.0}{0.5}{Varying}{0.5}{5}{fig:ex4}
As the mean of the suboptimal arms get closer to $\mu^*$, more exploration can be afforded due to the small difference in payoffs.
This effect is illustrated in figure~\ref{fig:ex4}, where the optimal \ob{} increases along with the mean of the bad arms.

\begin{figure}[htbp]
    \begin{gnuplot}[terminal=epslatex,terminaloptions=color]
    set style data lines
    set grid
    set xlabel "Observation noise"
    set ylabel "Total regret"
    set yrange [13:15]
    set xrange [0.1:0.4]
    plot "../data/bruteforce/good-5.0,0.5\_bad-1.5,0.5\_est-10.0,2.0\_num-5\_obnoise-(0.01,0.4,0.01)\_rounds-1000\_reps-100000.data" using (column(1)==1000?column(2):1/0):(5000-column(3)) title ""
\end{gnuplot}
\caption{Sample from~\ref{fig:ex4}, with mean 1.5. This is the result of averaging over 100,000 runs.}
\label{fig:ex5}
\end{figure}
\exsetup{5.0}{0.5}{1.5}{0.5}{5}{fig:ex5}

In figure~\ref{fig:ex6} we see that changing the standard deviation of the suboptimal arms does not influence the best choice of observation noise, except for when the number of rounds is very low.
In fact, this seems to hold as long as the standard deviation is kept within reasonable limits.

In contrast, the standard deviation of the best arm has a large influence on \ob{}. As it is increased, \ob{} also increases. (Figure~\ref{fig:ex7}).

\begin{figure}[hbtp]
    \centering
    \begin{gnuplot}[terminal=epslatex,terminaloptions=color]
    set style data lines
    set grid
    set pm3d
    set xlabel "[r]{\\shortstack{Bad arm \\\\ standard deviation}}"
    set ylabel "Rounds" rotate
    set zlabel "Observation noise" rotate
    set xyplane 0
    set xtics 0.5
    set ytics 200
    set cbtics 0.03
    set yrange [100:1000]
    set xrange [3.5:0]
    set zrange [0:0.6]
    set palette rgbformulae 23,28,3
    splot "../data/bandit/good-5.0,0.5\_bad-4.0,(0.1,3.5,0.2)\_est-10.0,2.0\_num-5\_rounds-1000\_algo-UCB1.data" using 4:(column(6)>99?column(6):1/0):7 with line lt -1 lw 1 title ""
    \end{gnuplot}
\caption{Varying standard deviation of the bad arms}
\label{fig:ex6}
\end{figure}
\exsetup{5.0}{0.5}{4.0}{Variable}{5}{fig:ex6}


\begin{figure}[hbtp]
    \centering
    \begin{gnuplot}[terminal=epslatex,terminaloptions=color]
    set grid
    set pm3d
    set ylabel "[r]{\\shortstack{Best arm \\\\ standard deviation}}"
    set xlabel "Rounds"
    set zlabel "Observation noise" rotate
    set xyplane 0
    set palette rgbformulae 23,28,3
    set view 65,312
    set xtics 200
    set ytics 1
    set ytics offset 0,-0.5
    splot "../data/bandit/good-6.5,(0.1,5.0,0.9)\_bad-4.0,0.1\_est-0.0,50.0\_num-5\_rounds-1000\_algo-UCB1.data" using 6:2:7 with line lt -1 lw 2 title ""
    \end{gnuplot}
\caption{Varying standard deviation of the best arm}
\label{fig:ex7}
\end{figure}
\exsetup{6.5}{Variable}{4.0}{0.1}{5}{fig:ex7}


\section{Goore Game}

