\chapter{Conclusion and further work}
\label{ch:conclusion}
\textit{Approx. 5 pages}

\section{Summary of Results}

\section{Conclusion}
\emph{“My conclusion offers a compelling final comment to my argument, one that is persuasive for my intended audience.”}

By doing thousands of simulations we have discovered the observation noises that maximise LTS performance in a multitude of settings for the multi armed bandit.
We have also shown how the best observation noise varies with different parameters, thereby providing an improved intuition towards how the observation noise affects performance.

We have also seen that the best observation noise is not trivially decreasing or increasing as we change simulation parameters?

In in goore game having more players only effects the optimal observation noise when you have a low white nose,
otherwise they all converge towards the same optimal observation noise. Changing the ratio effects the observation
noise, in that way that the closer it is to 100/0 in its distribution the higher is the optimal observation noise.
This also applies with more time the Goore game have, since with more time each player have more time to explore
what is the best arm to pick. White noise do increase the optimal observation noise, and this do seam to be at a
linear rate for each time step with bigger variations at lower time steps.

\section{Contributions}
Our contributions consist mainly of an improved empirical and intuitive understanding of how the performance of the LTS strategy can be maximised by choosing correct observation noise.


\section{Further Work}
Use Gaussian processes to create models to infirrer the optimal observation noise.

