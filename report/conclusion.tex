\chapter{Conclusion and further work}
\label{ch:conclusion}

\section{Summary of Results}

\section{Conclusion}

By doing thousands of simulations we have discovered the observation noises that maximise LTS performance in a multitude of settings for the multi armed bandit.
We have also shown how the best observation noise varies with different parameters, thereby providing an improved intuition towards how the observation noise affects performance.

We have also seen that the best observation noise is not trivially decreasing or increasing as we change simulation parameters?

In the Goore game having more players only affects the optimal observation noise when you have a low white nose,
otherwise they all converge towards the same optimal observation noise. By changing the ratio, it effects the observation
noise in the way when it is closer to 0 or 1 in ratio the optimal observation noise is higher.
Observation noise also rises when the players have more time on their hands, since with more time each player may use more time 
to explore and find the best arm to pick and exploit. With higher white noise,the optimal observation noise do 
increase, and this do seam to be at a linear rate for each time step, with bigger variations at lower time steps.

\section{Contributions}
Our contributions consist mainly of an improved empirical and intuitive understanding of how the performance of the LTS strategy can be maximised by choosing correct observation noise.


\section{Further Work}
Use Gaussian processes to create models or graphs that can be used to inferr the optimal observation noise for a realistic subset of possible parameter configurations for the multi-armed bandit problem and the Goore game. Meaning that the Gaussian processes will use existing data to estimate unknown optimal observation noises. Currently all optimal observation noises are discovered through a brute-force approach thus if anyone requires a non-existent configuration of parameters it would be necessary to run the program with these new parameters. By employing Gaussian processes the user would be able to estimate the optimal observation noise by using existing data.

