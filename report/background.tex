\chapter{Background}
\label{ch:background}

\section{The multi-armed bandit problem}
A popular and surprisingly fitting way to model stochastic optimization problems 
is through the multi-armed bandit scenario. Imagine, if you will, the iconic 
gambling machine to be found in any casino. Only this apparatus has not one, but 
many arms with which a player may try her luck. Oh, but which arm to choose? 
With so many possible choices, a strategy for optimising the machine’s payout 
must be employed. 

Before any arm-pulling has been done, nothing is known about the possible 
payoffs from the multitude of available arms exept that some arms are better 
than others -- obviously our player will profit the most from playing only the 
best arm. And so, it comes to this: she must make a tradeoff between exploration 
-- discovering which arm is truly the best one -- and exploitation -- actually 
picking the best arm.


\section{Learning automata}
The model of a learning automaton is illustrated in figure~\ref{fig:la}.
\begin{figure}[ht]
        \begin{verbatim}
                   |-------------|  r
              |--->| Environment |----|
              |    |-------------|    |
              |                       | 
              | a  |-------------|    |
              |----|   Learner   |<---|
                   |-------------|
         
                     a: action
                     r: reward
        \end{verbatim}
        \label{fig:la}
    \caption{Replace with drawing}
\end{figure}

\section{Thompson sampling}
Thompson sampling is a technique for updating prior knowledge with new 
information.

It employs the nefarious and infamous parameter known as the Observation Noise, 
or $\sigma_{ob}$. This is not for the faint of heart!


\section{The Goore Game}
Introduced by Tsetlin in 1973, the Goore game is a cooperation game where each  
player is not privvy to information on the other players choices. It can be 
described as a voting process where the referree knows the correct answer, and 
the referrers cast their votes as simple ‘yes’ or ‘no’. The correct answer is a 
certain number of yes’s and no’s, and players are rewarded based on how close 
they collectively are to the ideal solution.



\section{Gaussian process}
A secretive procedure invented by a certain Mr. G. Process. Contrary to 
widespread belief, it does not involve the use of black magick.

\section{Python}

\section{Pypy}

\section{Haskell}



% \section{Sample stuff}
% Some simple and useful latex formatting.
% 
% \subsection{Quotations and citing}
% It is explained in detail in \cite[Ch.20]{Norvig03} that
% 
% \begin{quotation}
% \noindent \textit{``the true hypothesis eventually dominates the Bayesian 
% predication. For any fixed prior that does not rule out the true hypothesis, the 
% posterior probability of any false hypothesis will eventually vanish, simply 
% because the probability of generating ``uncharacteristic'' data indefinitely is 
% vanishingly small.''}
% \end{quotation}
% 
% \subsection{Figures}
% This distribution, and its probability density function, is displayed in Figure 
% \ref{fig:gaussian_distr_pdf}.
% \begin{figure}[ht]
%     \center\includegraphics[width=10cm]{images/normal_distr_pdf}	
%     \label{fig:gaussian_distr_pdf}
%     \caption{The Normal distribution PDF.}
% \end{figure}
% 
% \subsection{Equations}
% By using these probabilities, and Bayes formula, we can derive the Bayes 
% classifier.
% \begin{equation}
%     P(\omega_i | \boldsymbol{x}, \mathcal{X}) = \frac{p(\boldsymbol{x}|\omega_i, 
%     \mathcal(X))P(\omega_i|\mathcal{X})}{\sum_{j=1}^{c}p(\boldsymbol{x}|\omega_j, 
%     \mathcal{X})P(\omega_j|\mathcal{X})},
%     \label{eq:bayes_formula_1}
% \end{equation}
% 
% when we can separate the training samples by class into $c$ subsets 
% $\mathcal{X}_1, \ldots, \mathcal{X}_c$, with the samples in $\mathcal{X}_i$ 
% belonging to $\omega_i$.
% 
