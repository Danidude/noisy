\chapter{Introduction}
\label{ch:introduction}

\section{Introduction}
Both the multi-armed bandit problem and the Goore game are settings which help illustrated the exploration-exploitation dilemma.
To succeed in these games the player must weigh the value of exploration against the value of exploitation, and the goal is to achieve the highest possible payoff.
Many performant strategies have been developed for both these games.
For the multi-armed bandit problem, methods employing what is called Thompson sampling have become increasingly popular in the recent years.
Thompson sampling was initially described by Thompson already in 1933 -- making it one of the first solution strategies for multi-armed bandits -- but it has not received much attention until recently.
As for the Goore game, of special interest is the type of solution where the players are modelled as playing on 2-armed gambling machines -- a special case of the multi-armed bandit problem.

In this paper we take a closer look on the use of Thompson sampling in the Gaussian version of the multi-armed bandit, and we then employ this model in the Goore game.
In the Gaussian bandits the reward distributions are normally distributed, and this causes the observation noise parameter \ob{} to arise when applying Thompson sampling to the problem.
This parameter is therefore the focus of this report.
We investigate empirically the optimal observation noise settings for different configurations in the multi-armed bandit problem and in the Goore game.

Multi-armed bandit approaches have multiple real life applications, and we will mention a few here.
It has been used to to manage the prioritising of multiple projects in large corporations, where money and manpower is distributed based on the expected return of investment.
As time passes resources are redistributed as it becomes more apparent which projects provide the highest profit.
In addition, multi-armed bandit approaches have been used in clinical trials and to solve scheduling problems in networks.

\section{Report outline}
The rest of this report is structured as follows: Chapter
\ref{ch:background} presents an overview of the background for our research.
Specifically, it describes in greater detail the multi-armed bandit problem, the Goore game and the
technique known as Thompson sampling. Aditionally, other algorithms are explained as Thompson sampling
have been compared to various other algorithms.

Subsequently, chapter \ref{ch:solution} describes the different solutions that were used, for both the Goore game
and the multi-armed banditproblem, and the reasoning behind the choices of solutions. Moreover, it provides 
alternative approaches and justifications for not selecting them. The chapter also covers what the project
and its results contributes to the field of research and why the project itself is original. 

Chapter \ref{ch:results} discusses the results. The Goore game and the multi-armed bandit problem have some
identical parameters and some distinct ones. The results are presented as graphs or tables. As there is a significant
amount of data and several parameters the focus is on why the optimal observation noise is optimal, how changes
in different parameters affects the observation noise and how Thompson sampling compares to other options.

Finally, chapter \ref{ch:conclusion} rounds off the paper by summarizing the problem, solution and result. It then goes
on to describe the new benefits the solution brings and why it is superior to existing solutions. Finally, it has a few 
suggestions for future work.
