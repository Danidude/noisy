\chapter{Introduction}
\label{ch:introduction}

\section{Introduction}
Both the multi-armed bandit problem and the Goore game are scenarioes which help illustrate the exploration-exploitation dilemma.
To succeed in these games the player must weigh the value of exploration against the value of exploitation, and the goal is to achieve the highest possible payoff.
Many performant strategies have been developed for both these games.
For the multi-armed bandit problem, methods employing what is called Thompson sampling \cite{Thompson1933} have become increasingly popular in the recent years.
Thompson sampling was initially described by Thompson already in 1933 -- making it one of the first solution strategies for multi-armed bandits.
As for the Goore game, of special interest is the type of solution where the players are modelled as playing on 2-armed gambling machines -- a special case of the multi-armed bandit problem.

In this paper we take a closer look on the use of Thompson sampling in the Gaussian version of the multi-armed bandit, and we then employ this model in the Goore game.
In the Gaussian bandits the reward distributions are normally distributed, and this causes the observation noise parameter \ob{} to arise when applying Thompson sampling to the problem.
This parameter is therefore the focus of this report.
We investigate empirically the optimal observation noise settings for different configurations in the multi-armed bandit problem and in the Goore game.

Multi-armed bandit approaches have multiple real life applications, and we will mention a few here.
It has been used to to manage the prioritising of multiple projects in large corporations, where money and manpower is distributed based on the expected return of investment \cite{Glimsdal12}.
As time passes resources are redistributed as it becomes more apparent which projects provide the highest profit.
In addition, multi-armed bandit approaches have been used in clinical trials, to solve scheduling problems in networks, in online advertising, news article selection and wireless networks \cite{Granmo07, Glimsdal12}.

\section{Report outline}
The rest of this report is structured as follows: Chapter~\ref{ch:background} presents an overview of the background for our research.
It describes the multi-armed bandit problem, the Goore game and the Thompson sampling technique, in addition to the UCB1 algorithm, another well-known strategy for the multi-armed bandit problem with which we later compare Thompson sampling.
Next, chapter~\ref{ch:solution} outlines the ways in which we performed our investigations of both the multi-armed bandit problem and the Goore game.
After that chapter~\ref{ch:results} discusses the results we have obtained, illustrated with graphs.
Here the focus is on gaining an intuitive understanding of the results, rather than expounding on mathematical peculiarities.
Finally, chapter~\ref{ch:conclusion} rounds off the paper by summarising the problem, solution and result, before suggesting ideas for future work.
