\chapter{Introduction}
\label{ch:introduction}

\section{Introduction}
The purpose of this project was to determine the optimal observation noise for different 
configurations in the multi-armed bandit problem and in the Goore game using local
Thompson sampling. The multi-armed bandit problem gives a gambler the option to pull one
out of several arms. Each arm will return a reward based on a distribution that may or may not be
unique to that arm. The gambler will have to decide which arm is the better one to maximize his
reward.

The Goore game builds on the same principles. However in this game there are several players
but exactly two arms for each player to chose from. To maximize the reward the different
players need to obtain a certian distribution of arms pulled, for example 70 percent of the players chose
arm number one while the remaining 30 percent chose arm number 2. Thompson sampling aims to solve
this problem by chosing an armed based on its probability of returning the highest reward. William R.
Thompson wrote about Thompson sampling in an article from 1933 and it has not seen much use since.

Multi-armed bandit approaches have real life applications as they have been used to manage multiple
projects in large corporations. Money and manpower would be distributed based on the expected
return of investment. As time passed resources could then be redistributed as it became more apperant
which projects were the lucerative ones. In addition, multi-armed bandit approaches have been used
in clinical trials and to solve scheduling problems.

\section{Problem Solution}
The goal of the project was to evaluate and map the observation noise to its expected performance using
a Gaussian Process; and using this information to create a model for inferring the optimal observation noise 
for a realistic subset of possible configurations. The problem definition was faily open throughout the entire
project period and there were always alternative solutions to chose from. 

Originally the project definition asserted that the solution would use Gaussian Processes it quickly got down-prioritized.
The focus was shifted towards the core of the problem, determining the optimal observation noise. While
the project initially were limited to Thompson sampling we briefly investigated other algorithms to compare
and contrast. 

\section{Report outline}
The rest of this report is structured as follows: Chapter
\ref{ch:background} presents an overview of the background for our research.
Specifically, it describes in greater detail the multi-armed bandit problem, the Goore game and the
technique known as Thompson sampling. Aditionally, other algorithms are explained as Thompson sampling
have been compared to various other algorithms.

Subsequently, chapter \ref{ch:solution} describes the different solutions that were used, for both the Goore game
and the multi-armed banditproblem, and the reasoning behind the choices of solutions. Moreover, it provides 
alternative approaches and justifications for not selecting them. The chapter also covers what the project
and its results contributes to the field of research and why the project itself is original. 

Chapter \ref{ch:results} discusses the results. The Goore game and the multi-armed bandit problem have some
identical parameters and some distinct ones. The results are presented as graphs or tables. As there is a significant
amount of data and several parameters the focus is on why the optimal observation noise is optimal, how changes
in different parameters affects the observation noise and how Thompson sampling compares to other options.

Finally, chapter \ref{ch:conclusion} rounds off the paper by summarizing the problem, solution and result. It then goes
on to describe the new benefits the solution brings and why it is superior to existing solutions. Finally, it has a few 
suggestions for future work.
