\chapter{Solution}
\label{ch:solution}

The solution consists of several parts, as we have worked on both a single multi-armed bandit scenario, and multiple bandits playing the Goore Game.
The solution produces machine-readable result files suitable for input to a programme which can infer the best observation noise to use through the use of a Gaussian process.

\section{Multi-armed bandit}
The multi-armed bandits in our trials have one uniquely best arm, and all other arms are identically distributed.
Following is a list of parameters used for the multi-armed bandits.
\begin{itemize}
    \item Mean of the best arm
    \item Standard deviation of the best arm
    \item Mean of the bad arms
    \item Standard deviation of the bad arms
    \item Number of arms
    \item Estimate mean
    \item Estimate standard deviation
    \item Observation noise
\end{itemize}

The means are kept in (0,10), and the standard deviation on all arms is kept the same value well below the mean.
The number of arms we test lies between 2 and 50.
Estimate mean and standard deviation are specific for the local Thompson sampling strategy, and are kept at twice the mean of the best arm and 2, respectively.
We calculate the best observation noise for a range of different timesteps up to 1000.

\subsection{Solution strategy}
We developed a couple of ways to compute and visualise the needed data.
A brute force method calculating cumulative rewards for all given observation noises, and a method that, given the simulation parameters, employs a bandit approach to calculating the best observation noise.

In addition we calculate instant rewards at log-scale timesteps given the simulation parameters, for verifying that the solution found is in fact the best one.

\subsubsection{Brute force solution}
By testing all the values in a given range it is possible to determine where the maximum lies.
Although this method is crude, it is helpful for gaining an indication towards what values are best or what values need more testing.
The downside of this solution is that many tests and much computation may be required, and most of the results generated are not interesting if we in reality are only looking for a (unique) maximum.

\subsubsection{Bandit approach}
As the multi-armed bandit approach can be used to find the maximum of a
stochastic function, it fits perfectly to the task of finding the best
observation noise for a multi-armed bandit employing the local Thompson
sampling strategy.

\subsubsection{Measuring performance}
There are several ways to keep score and look at data from a bandit simulation.
First, one can look at the cumulative reward recieved by the player.
Second, there is the instant rewards, that is, the reward received by the player at a given time $t$.
By logging this one can easily see how the player increases its performance over time.
Third, a much used approach is to calculate regret.
In fact, the multi-armed bandit problem is often stated as a minimisation problem with respect to regret.
Last, the number of times the best arm is selected at any given time can be an interesting metric.

\section{Goore Game}
Contrary to the multi-armed bandit scenario the bandits in the Goore game have identical starting
values on both their arms. The way we measured performance was by looking at the reward for each observation
noise. By doing so we would find the highest reward, and the observation noise used by that simulation, witch
is the best observation noise for that instance. Following is a list of all parameters used for the Goore game:
\begin{itemize}
\item Initial mean of both arms
\item Initial standard deviation of both arms
\item Number of bandits
\item Ratio for the selected arms(Yes and no)
\item Gaussian white noise
\item Number of rounds the game lasted
\end{itemize}

The mean and standard deviation on the arms are fairly insignificant as long as the mean or the standard deviation is high enough to ensure that each arm is chosen atleast once. We tested between 2 and 40 bandits to understand how the increase in players affected the observation noise. Additionally we changed the ratio to different values to see whether certain target values were more difficult hit. We only used ratios below 0.5, since the ratio is symmetric above and below 0.5 in ratio. For example is 0.2 the same as 0.8 in ratio, just different arms arms are picked, and both are identical at the beginning. Finally by varying the Gaussian white noise we observed how the higher standard deviation in reward affected the players.


%By testing all the values in a given range to determine where the maximum lies under different
%circumstances. A downside to this menthod is that it will generate some use full data and some not so
%use full data. For example, on a ratio 50/50, with 5 players is it hard for the  odd player to 
%determine whether yes or no is the best answer. However this method do gives some use full inn-site on
%how Goore Game works under strange circumstances that are nod ideal. As this solution is easy to set
%up and start, it is however needed to take many more tests and more time.


% \section{Proposed solution / algorithm}
% 
% \subsection{The basic algorithm}
% 
% \subsection{Discussion of design issues}
% 
% 
% \subsection{Algorithmic Enhancements}
% 
% 
% \subsection{Discussion of the Parameter Space}
% 
% 
% \section{Prototype}
% 
% \section{Justification of Claim to Originality}
% 
% \section{Valuation of Contribution}
% 
% \section{Alternatives}
